\documentclass[12pt]{article}
\usepackage{geometry}
\geometry{letterpaper}
\usepackage{graphicx}
% \usepackage{grffile}
% \usepackage{caption}
% \usepackage{pdflscape}
%\usepackage{verbatim}

% \renewcommand{\figurename}{Figure}


\title{Draft plots for trends in ecology and conservation}
\author{}
% \date{}

% \setlength\parskip{0.1in}
% \setlength\parindent{0in}

% \thispagestyle{empty}
% \pagenumbering{gobble}

\topmargin -2cm
\oddsidemargin -1.0cm
\evensidemargin -1.0cm
\textwidth 18.5cm
\textheight 23cm

\begin{document}

\maketitle

% \clearpage

\begin{figure}[htbp]
\centering
\includegraphics[width=6in]{../figs/decades-and-booms-viridis.pdf}
\caption{
(A--D) Top 9 1-gram nouns and 2-gram nouns/adjectives from the 1940s and 2000s.
Labels refer to ``lemmas'' or root words. For example, plant and plants as
n-grams are both combined and labelled as ``plant''. (E, F) Most rapidly
increasing 1-gram nouns and 2-gram nouns/adjectives from 1980--2010.}
\label{fig:decades}
\end{figure}

\begin{figure}[htbp]
\centering
\includegraphics[width=6in]{../figs/blanks-viridis2.pdf}
\caption{Top 8 $<$fill in the blank$>$ followed by the term in the top left
corner where the terms are either nouns or adjectives. The top n-grams are
defined based on summing the occurrences throughout the entire time period
thereby weighting recent years more heavily.}
\label{fig:top}
\end{figure}

\clearpage

\section{Supporting Information}

\renewcommand{\thefigure}{S\arabic{figure}}
\renewcommand{\thetable}{S\arabic{table}}
\setcounter{figure}{0}
\setcounter{table}{0}

\begin{figure}[htbp]
\centering
\includegraphics[width=\textwidth]{../figs/booms.pdf}
\caption{Most rapidly declining (A, B) and increasing (C, D) 1-gram nouns and
  2-gram nouns/adjectives. The grey shaded region indicates the time period
  during which the average rate of change is calculated. The terms had to be
  within the top 300 terms in the 1940s or the 2000s and have at least one
  occurrence in 20 of the 30 years considered for the rate of change (1930--1960
  or 1980--2010). Panels A and B are duplicated from Fig.~\ref{fig:decades}}
\label{fig:booms}
\end{figure}

% \clearpage
%
% \section*{Drafts of hand-picked examples on various themes}
%
% The following are some draft panels on various themes. They are by no means
% complete or final. There are a number of themes that we have not worked on yet and much to still explore within these themes.
% This is the major area where we could use your help! I have an R Shiny app that
% I will pass on so you can interactively play with the data to develop panels.
% The final figures will just include the most interesting panels with the rest in the supplementary material. That's all the rage these days (Figure~\ref{fig:methods}F).
%
% Some of these labels represent a number of underlying n-grams based on alternative acronyms or singular/plural versions. That's not documented yet.
%
% \begin{figure}[htbp]
% \centering
% \includegraphics[width=\textwidth]{../figs/stats1.pdf}
% \caption{Methods \ldots (so far just statistical)}
% \label{fig:methods}
% \end{figure}
%
% \clearpage
%
% \begin{figure}[htbp]
% \centering
% \includegraphics[width=0.79\textwidth]{../figs/scale-panels-2-cont.pdf}
% \caption{Scale and all sorts of other ecological concepts  \ldots}
% \end{figure}
%
% \clearpage
%
% \begin{figure}[htbp]
% \centering
% \includegraphics[width=\textwidth]{../figs/conservation-panels-3.pdf}
% \caption{Conservation \ldots}
% \end{figure}
%
% \clearpage
%
% \begin{figure}[htbp]
% \centering
% \includegraphics[width=\textwidth]{../figs/human-impacts-panels-4.pdf}
% \caption{Human impacts \ldots}
% \end{figure}
%
\end{document}
